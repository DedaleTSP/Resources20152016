% Generated by Sphinx.
\def\sphinxdocclass{report}
\documentclass[letterpaper,10pt,french]{sphinxmanual}

\usepackage[utf8]{inputenc}
\ifdefined\DeclareUnicodeCharacter
  \DeclareUnicodeCharacter{00A0}{\nobreakspace}
\else\fi
\usepackage{cmap}
\usepackage[T1]{fontenc}
\usepackage{amsmath,amssymb}
\usepackage{babel}
\usepackage{times}
\usepackage[Sonny]{fncychap}
\usepackage{longtable}
\usepackage{sphinx}
\usepackage{multirow}
\usepackage{eqparbox}


\addto\captionsfrench{\renewcommand{\figurename}{Fig. }}
\addto\captionsfrench{\renewcommand{\tablename}{Tableau }}
\SetupFloatingEnvironment{literal-block}{name=Code source }

\addto\extrasfrench{\def\pageautorefname{page}}

\setcounter{tocdepth}{1}


\title{Projet Dédale 2015-2016 Documentation}
\date{mai 31, 2016}
\release{0.0.1a}
\author{Team Dédale 2015-2016}
\newcommand{\sphinxlogo}{}
\renewcommand{\releasename}{Version}
\makeindex

\makeatletter
\def\PYG@reset{\let\PYG@it=\relax \let\PYG@bf=\relax%
    \let\PYG@ul=\relax \let\PYG@tc=\relax%
    \let\PYG@bc=\relax \let\PYG@ff=\relax}
\def\PYG@tok#1{\csname PYG@tok@#1\endcsname}
\def\PYG@toks#1+{\ifx\relax#1\empty\else%
    \PYG@tok{#1}\expandafter\PYG@toks\fi}
\def\PYG@do#1{\PYG@bc{\PYG@tc{\PYG@ul{%
    \PYG@it{\PYG@bf{\PYG@ff{#1}}}}}}}
\def\PYG#1#2{\PYG@reset\PYG@toks#1+\relax+\PYG@do{#2}}

\expandafter\def\csname PYG@tok@mi\endcsname{\def\PYG@tc##1{\textcolor[rgb]{0.13,0.50,0.31}{##1}}}
\expandafter\def\csname PYG@tok@kc\endcsname{\let\PYG@bf=\textbf\def\PYG@tc##1{\textcolor[rgb]{0.00,0.44,0.13}{##1}}}
\expandafter\def\csname PYG@tok@se\endcsname{\let\PYG@bf=\textbf\def\PYG@tc##1{\textcolor[rgb]{0.25,0.44,0.63}{##1}}}
\expandafter\def\csname PYG@tok@s1\endcsname{\def\PYG@tc##1{\textcolor[rgb]{0.25,0.44,0.63}{##1}}}
\expandafter\def\csname PYG@tok@sd\endcsname{\let\PYG@it=\textit\def\PYG@tc##1{\textcolor[rgb]{0.25,0.44,0.63}{##1}}}
\expandafter\def\csname PYG@tok@sh\endcsname{\def\PYG@tc##1{\textcolor[rgb]{0.25,0.44,0.63}{##1}}}
\expandafter\def\csname PYG@tok@go\endcsname{\def\PYG@tc##1{\textcolor[rgb]{0.20,0.20,0.20}{##1}}}
\expandafter\def\csname PYG@tok@sr\endcsname{\def\PYG@tc##1{\textcolor[rgb]{0.14,0.33,0.53}{##1}}}
\expandafter\def\csname PYG@tok@cp\endcsname{\def\PYG@tc##1{\textcolor[rgb]{0.00,0.44,0.13}{##1}}}
\expandafter\def\csname PYG@tok@w\endcsname{\def\PYG@tc##1{\textcolor[rgb]{0.73,0.73,0.73}{##1}}}
\expandafter\def\csname PYG@tok@il\endcsname{\def\PYG@tc##1{\textcolor[rgb]{0.13,0.50,0.31}{##1}}}
\expandafter\def\csname PYG@tok@ch\endcsname{\let\PYG@it=\textit\def\PYG@tc##1{\textcolor[rgb]{0.25,0.50,0.56}{##1}}}
\expandafter\def\csname PYG@tok@nn\endcsname{\let\PYG@bf=\textbf\def\PYG@tc##1{\textcolor[rgb]{0.05,0.52,0.71}{##1}}}
\expandafter\def\csname PYG@tok@no\endcsname{\def\PYG@tc##1{\textcolor[rgb]{0.38,0.68,0.84}{##1}}}
\expandafter\def\csname PYG@tok@kn\endcsname{\let\PYG@bf=\textbf\def\PYG@tc##1{\textcolor[rgb]{0.00,0.44,0.13}{##1}}}
\expandafter\def\csname PYG@tok@nv\endcsname{\def\PYG@tc##1{\textcolor[rgb]{0.73,0.38,0.84}{##1}}}
\expandafter\def\csname PYG@tok@cs\endcsname{\def\PYG@tc##1{\textcolor[rgb]{0.25,0.50,0.56}{##1}}\def\PYG@bc##1{\setlength{\fboxsep}{0pt}\colorbox[rgb]{1.00,0.94,0.94}{\strut ##1}}}
\expandafter\def\csname PYG@tok@na\endcsname{\def\PYG@tc##1{\textcolor[rgb]{0.25,0.44,0.63}{##1}}}
\expandafter\def\csname PYG@tok@err\endcsname{\def\PYG@bc##1{\setlength{\fboxsep}{0pt}\fcolorbox[rgb]{1.00,0.00,0.00}{1,1,1}{\strut ##1}}}
\expandafter\def\csname PYG@tok@c1\endcsname{\let\PYG@it=\textit\def\PYG@tc##1{\textcolor[rgb]{0.25,0.50,0.56}{##1}}}
\expandafter\def\csname PYG@tok@c\endcsname{\let\PYG@it=\textit\def\PYG@tc##1{\textcolor[rgb]{0.25,0.50,0.56}{##1}}}
\expandafter\def\csname PYG@tok@nb\endcsname{\def\PYG@tc##1{\textcolor[rgb]{0.00,0.44,0.13}{##1}}}
\expandafter\def\csname PYG@tok@ss\endcsname{\def\PYG@tc##1{\textcolor[rgb]{0.32,0.47,0.09}{##1}}}
\expandafter\def\csname PYG@tok@ne\endcsname{\def\PYG@tc##1{\textcolor[rgb]{0.00,0.44,0.13}{##1}}}
\expandafter\def\csname PYG@tok@sx\endcsname{\def\PYG@tc##1{\textcolor[rgb]{0.78,0.36,0.04}{##1}}}
\expandafter\def\csname PYG@tok@vi\endcsname{\def\PYG@tc##1{\textcolor[rgb]{0.73,0.38,0.84}{##1}}}
\expandafter\def\csname PYG@tok@gr\endcsname{\def\PYG@tc##1{\textcolor[rgb]{1.00,0.00,0.00}{##1}}}
\expandafter\def\csname PYG@tok@gi\endcsname{\def\PYG@tc##1{\textcolor[rgb]{0.00,0.63,0.00}{##1}}}
\expandafter\def\csname PYG@tok@sb\endcsname{\def\PYG@tc##1{\textcolor[rgb]{0.25,0.44,0.63}{##1}}}
\expandafter\def\csname PYG@tok@ow\endcsname{\let\PYG@bf=\textbf\def\PYG@tc##1{\textcolor[rgb]{0.00,0.44,0.13}{##1}}}
\expandafter\def\csname PYG@tok@gu\endcsname{\let\PYG@bf=\textbf\def\PYG@tc##1{\textcolor[rgb]{0.50,0.00,0.50}{##1}}}
\expandafter\def\csname PYG@tok@gt\endcsname{\def\PYG@tc##1{\textcolor[rgb]{0.00,0.27,0.87}{##1}}}
\expandafter\def\csname PYG@tok@gp\endcsname{\let\PYG@bf=\textbf\def\PYG@tc##1{\textcolor[rgb]{0.78,0.36,0.04}{##1}}}
\expandafter\def\csname PYG@tok@m\endcsname{\def\PYG@tc##1{\textcolor[rgb]{0.13,0.50,0.31}{##1}}}
\expandafter\def\csname PYG@tok@nf\endcsname{\def\PYG@tc##1{\textcolor[rgb]{0.02,0.16,0.49}{##1}}}
\expandafter\def\csname PYG@tok@si\endcsname{\let\PYG@it=\textit\def\PYG@tc##1{\textcolor[rgb]{0.44,0.63,0.82}{##1}}}
\expandafter\def\csname PYG@tok@gd\endcsname{\def\PYG@tc##1{\textcolor[rgb]{0.63,0.00,0.00}{##1}}}
\expandafter\def\csname PYG@tok@kp\endcsname{\def\PYG@tc##1{\textcolor[rgb]{0.00,0.44,0.13}{##1}}}
\expandafter\def\csname PYG@tok@cpf\endcsname{\let\PYG@it=\textit\def\PYG@tc##1{\textcolor[rgb]{0.25,0.50,0.56}{##1}}}
\expandafter\def\csname PYG@tok@mb\endcsname{\def\PYG@tc##1{\textcolor[rgb]{0.13,0.50,0.31}{##1}}}
\expandafter\def\csname PYG@tok@kt\endcsname{\def\PYG@tc##1{\textcolor[rgb]{0.56,0.13,0.00}{##1}}}
\expandafter\def\csname PYG@tok@k\endcsname{\let\PYG@bf=\textbf\def\PYG@tc##1{\textcolor[rgb]{0.00,0.44,0.13}{##1}}}
\expandafter\def\csname PYG@tok@vc\endcsname{\def\PYG@tc##1{\textcolor[rgb]{0.73,0.38,0.84}{##1}}}
\expandafter\def\csname PYG@tok@mo\endcsname{\def\PYG@tc##1{\textcolor[rgb]{0.13,0.50,0.31}{##1}}}
\expandafter\def\csname PYG@tok@s2\endcsname{\def\PYG@tc##1{\textcolor[rgb]{0.25,0.44,0.63}{##1}}}
\expandafter\def\csname PYG@tok@gs\endcsname{\let\PYG@bf=\textbf}
\expandafter\def\csname PYG@tok@kr\endcsname{\let\PYG@bf=\textbf\def\PYG@tc##1{\textcolor[rgb]{0.00,0.44,0.13}{##1}}}
\expandafter\def\csname PYG@tok@vg\endcsname{\def\PYG@tc##1{\textcolor[rgb]{0.73,0.38,0.84}{##1}}}
\expandafter\def\csname PYG@tok@ge\endcsname{\let\PYG@it=\textit}
\expandafter\def\csname PYG@tok@kd\endcsname{\let\PYG@bf=\textbf\def\PYG@tc##1{\textcolor[rgb]{0.00,0.44,0.13}{##1}}}
\expandafter\def\csname PYG@tok@cm\endcsname{\let\PYG@it=\textit\def\PYG@tc##1{\textcolor[rgb]{0.25,0.50,0.56}{##1}}}
\expandafter\def\csname PYG@tok@s\endcsname{\def\PYG@tc##1{\textcolor[rgb]{0.25,0.44,0.63}{##1}}}
\expandafter\def\csname PYG@tok@sc\endcsname{\def\PYG@tc##1{\textcolor[rgb]{0.25,0.44,0.63}{##1}}}
\expandafter\def\csname PYG@tok@nl\endcsname{\let\PYG@bf=\textbf\def\PYG@tc##1{\textcolor[rgb]{0.00,0.13,0.44}{##1}}}
\expandafter\def\csname PYG@tok@mh\endcsname{\def\PYG@tc##1{\textcolor[rgb]{0.13,0.50,0.31}{##1}}}
\expandafter\def\csname PYG@tok@nc\endcsname{\let\PYG@bf=\textbf\def\PYG@tc##1{\textcolor[rgb]{0.05,0.52,0.71}{##1}}}
\expandafter\def\csname PYG@tok@bp\endcsname{\def\PYG@tc##1{\textcolor[rgb]{0.00,0.44,0.13}{##1}}}
\expandafter\def\csname PYG@tok@nd\endcsname{\let\PYG@bf=\textbf\def\PYG@tc##1{\textcolor[rgb]{0.33,0.33,0.33}{##1}}}
\expandafter\def\csname PYG@tok@nt\endcsname{\let\PYG@bf=\textbf\def\PYG@tc##1{\textcolor[rgb]{0.02,0.16,0.45}{##1}}}
\expandafter\def\csname PYG@tok@o\endcsname{\def\PYG@tc##1{\textcolor[rgb]{0.40,0.40,0.40}{##1}}}
\expandafter\def\csname PYG@tok@gh\endcsname{\let\PYG@bf=\textbf\def\PYG@tc##1{\textcolor[rgb]{0.00,0.00,0.50}{##1}}}
\expandafter\def\csname PYG@tok@ni\endcsname{\let\PYG@bf=\textbf\def\PYG@tc##1{\textcolor[rgb]{0.84,0.33,0.22}{##1}}}
\expandafter\def\csname PYG@tok@mf\endcsname{\def\PYG@tc##1{\textcolor[rgb]{0.13,0.50,0.31}{##1}}}

\def\PYGZbs{\char`\\}
\def\PYGZus{\char`\_}
\def\PYGZob{\char`\{}
\def\PYGZcb{\char`\}}
\def\PYGZca{\char`\^}
\def\PYGZam{\char`\&}
\def\PYGZlt{\char`\<}
\def\PYGZgt{\char`\>}
\def\PYGZsh{\char`\#}
\def\PYGZpc{\char`\%}
\def\PYGZdl{\char`\$}
\def\PYGZhy{\char`\-}
\def\PYGZsq{\char`\'}
\def\PYGZdq{\char`\"}
\def\PYGZti{\char`\~}
% for compatibility with earlier versions
\def\PYGZat{@}
\def\PYGZlb{[}
\def\PYGZrb{]}
\makeatother

\renewcommand\PYGZsq{\textquotesingle}

\begin{document}

\maketitle
\tableofcontents
\phantomsection\label{index::doc}


Cette documentation a pour but de rendre compte des connaissances acquises lors de l'année 2015-2016 autour du projet
Dédale, et de fournir un condensé des informations et ressources utiles à la continuation du projet.

\textbf{Table des matières:}


\chapter{Ardrone 2.0}
\label{ardrone::doc}\label{ardrone:ardrone-2-0}\label{ardrone:ressources-du-projet-dedale-2015-2016}

\section{Description générale}
\label{ardrone:description-generale}
Le projet utilise un drone construit par Parrot. Cette année nous avons travaillé avec l'Ardrone 2.0.
Il embarque 2 caméra (devant et dessous).


\section{Mise en place de la connexion}
\label{ardrone:mise-en-place-de-la-connexion}
Ce modèle de drone créer un réseau wifi ouvert de type ad-hoc. La première personne connectée dessus peut contrôler le drone.
Le réseau est accéssible très facilement puisque ouvert.


\section{Commandes AT}
\label{ardrone:commandes-at}
Le drone est piloté par des commandes AT. Un descriptif de ces commandes à été réalisé et est disponible dans le dossier
\code{ardrone} sur le dépot git \href{https://github.com/DedaleTSP/Resources20152016}{Ici}


\section{Informations utiles}
\label{ardrone:informations-utiles}
Une seule des deux caméra peut fonctionner à la fois. Pour couper les moteurs du drone en urgence, il suffit de
le mettre la tête en bas.


\section{Liens externes}
\label{ardrone:liens-externes}
Site Web d'un ancien ayant travaillé sur le drone : \href{http://www.upsilonaudio.com/category/drones/}{Upsilon Audio}
Plusieurs programmes et tuto sur le drone y sont disponibles.


\chapter{Raspberry Pi}
\label{raspberrypi:raspberry-pi}\label{raspberrypi::doc}

\section{Utilité}
\label{raspberrypi:utilite}
La raspberry est l'ordinateur embarqué sur le drone. Elle s'y connecte et intéragit avec permettant le pilotage du
drone par un programme de façon autonome (sans station de base au sol). Elle est aussi utiliser pour étendre la portée
de connexion au drone. En effet, la raspberry pi est connectée sur le wifi du drone et sur le wifi du campus grâce à
ses deux interfaces, et donc accessible par un opérateur depuis le wifi du campus. Concrètement, il est possible d'agir
sur le drone dès lors que l'on est connecté au wifi du campus par le biais de la raspberry pi.


\section{Installation}
\label{raspberrypi:installation}
Le système de la raspberry est contenu sur une carte SD. Cette dernière est branchée dans le slot prévu à cet effet sur
la rapsberry pi. Les principaux systèmes disponibles sont : \code{Raspbian} ; \code{ArchLinux-ARM} ; \code{Ubuntu-ARM}
(uniquement raspberry 2 et +). Le framework ROS utilisé pour le projet est disponible pour les trois précédents systèmes,
mais les paquets binaires uniquement sous \code{Ubuntu-ARM}. Ainsi, dans le cas de \code{Raspbian} et \code{ArchLinux-ARM},
le framework ROS et tous les paquets associés doivent être compilés depuis les sources (et ce n'est pas une bonne idée,
les sources de beaucoup de paquets ROS ne sont plus maintenues après la release du paquet binaire sur Ubuntu). En bref,
le seule système officiellement supportée par ROS est Ubuntu, vouloir en utiliser un autre n'implique qu'une perte de
temps. (Il est même plus rapide de mettre en place une machine virtuelle sous Ubuntu et d'y installer ROS, que de vouloir
l'installer sur un autre système où il n'est pas nativement supporté). Pour en revenir à l'installation d'un système sur
raspberry pi, une fois en possession d'une carte SD et d'une image système pour le modèle de raspberry pi, il suffit
de faire un :
\begin{quote}

\begin{Verbatim}[commandchars=\\\{\}]
\PYGZdl{} dd \PYG{k}{if}\PYG{o}{=}\PYG{l+s+s2}{\PYGZdq{}/chemin/vers/fichier/image\PYGZdq{}} \PYG{n+nv}{of}\PYG{o}{=}\PYG{l+s+s2}{\PYGZdq{}/dev/sdX\PYGZdq{}}
\end{Verbatim}
\end{quote}

Avec \code{/dev/sdX} le chemin vers la carte SD. Pour obtenir ce chemin, la commande \code{lsblk} est utile.
Pour plus d'informations, voir les pages \code{man} des différentes commandes utilisées, et la documentation officielle
des raspberry pi : \href{https://www.raspberrypi.org/documentation/installation/installing-images/linux.md}{Installer un système sur sa raspberry pi}


\section{Usage}
\label{raspberrypi:usage}
Pour accéder à la raspberry pi depuis un autre ordinateur, l'interface la plus utilisé est le \code{ssh}.
Elle est accessible dès lors que la raspberry et l'ordinateur sont en réseau, et ouvre un terminal distant sur la
raspberry depuis l'ordinateur, permettant pleinement de s'en servir. Il suffit, depuis un terminal, de lancer la commande :
\begin{quote}

\begin{Verbatim}[commandchars=\\\{\}]
\PYGZdl{} ssh \PYG{o}{[}user\PYG{o}{]}@\PYG{o}{[}hostnameORIPadress\PYG{o}{]}
\end{Verbatim}
\end{quote}

Dans le cas de la raspberry pi, l'utilisateur par défaut est \code{pi}, le nom d'hôte \code{raspberrypi} et l'addresse IP
varie selon le sous réseau dans lequel les deux appareils sont. Le mot de passe par défaut pour l'utilisateur \code{pi} est
\code{raspberry}. Un exemple concret serait (si la raspberry a 192.168.1.2 comme addresse IP) :
\begin{quote}

\begin{Verbatim}[commandchars=\\\{\}]
\PYGZdl{} ssh pi@192.168.1.2
\end{Verbatim}
\end{quote}

Pour plus d'informations, \code{man ssh}.


\section{Configuration}
\label{raspberrypi:configuration}
Un certain nombre de configuration ont été faites sur la raspberry pi afin de faciliter son utilisation.


\subsection{Démarrage}
\label{raspberrypi:demarrage}
L'utilisation de scripts de démarrage a été très utile dans notre projet, notamment un script permettant au démarrage
du système, d'envoyer un email contenant l'adresse IP de la raspberry pi (l'adresse pouvant changer à chaque redémarrages).
Il est possible de lancer un script au démarrage via la crontab. Elle est éditable en utilisant la commande :
\begin{quote}

\begin{Verbatim}[commandchars=\\\{\}]
\PYGZdl{} crontab \PYGZhy{}e
\end{Verbatim}
\end{quote}

et voici un exemple d'entrée dans la crontab :
\begin{quote}

\begin{Verbatim}[commandchars=\\\{\}]
@reboot /home/pi/Dedale201516\PYGZus{}RPI\PYGZus{}Setup/startup.sh
\end{Verbatim}
\end{quote}

qui lance le script \code{startup.sh} ayant pour code :
\begin{quote}

\begin{Verbatim}[commandchars=\\\{\},numbers=left,firstnumber=1,stepnumber=1]
\PYG{c+ch}{\PYGZsh{}!/bin/sh}
wget \PYGZhy{}\PYGZhy{}spider www.google.fr
\PYG{k}{while} \PYG{o}{[} \PYG{l+s+s2}{\PYGZdq{}}\PYG{n+nv}{\PYGZdl{}?}\PYG{l+s+s2}{\PYGZdq{}} !\PYG{o}{=} \PYG{l+m}{0} \PYG{o}{]} \PYG{p}{;} \PYG{k}{do}
        sleep 2
        wget \PYGZhy{}\PYGZhy{}spider www.google.fr
\PYG{k}{done}
\PYG{k}{while} \PYG{o}{[} \PYG{l+s+sb}{{}`}cat /sys/class/net/wlan1/operstate\PYG{l+s+sb}{{}`} !\PYG{o}{=} \PYG{l+s+s2}{\PYGZdq{}up\PYGZdq{}} \PYG{o}{]} \PYG{p}{;} \PYG{k}{do}
        sleep 2
\PYG{k}{done}
ip addr show wlan1 \PYG{p}{\textbar{}} mail \PYGZhy{}s \PYG{l+s+s2}{\PYGZdq{}RaspberryIP\PYGZdq{}} votreadresse@mail.com
\end{Verbatim}
\end{quote}

Ce script essaye simplement de se connneter à google, et dès qu'il réussie, vérifie que l'interface wifi est bien
disponible, puis envoie par email sa configuration wifi vers l'adresse \code{votreadresse@mail.com}


\subsection{Réseau}
\label{raspberrypi:reseau}
La configuration réseau est certainement la plus importante, elle permet à la raspberry de communiquer avec le drone et
de se connecter au wifi du campus. Elle se fait via le fichier \code{/etc/network/interfaces}. Nous utilisions ce fichier :
\begin{quote}

\begin{Verbatim}[commandchars=\\\{\}]
\PYGZsh{} Please note that this file is written to be used with dhcpcd.
\PYGZsh{} For static IP, consult /etc/dhcpcd.conf and \PYGZsq{}man dhcpcd.conf\PYGZsq{}.

auto lo
iface lo inet loopback

auto eth0
allow\PYGZhy{}hotplug eth0
iface eth0 inet static
    address 169.254.0.2
    netmask 255.255.255.0
    network 169.254.0.0
    broadcast 169.254.0.255

auto wlan0
allow\PYGZhy{}hotplug wlan0
\PYGZsh{}iface wlan0 inet dhcp
\PYGZsh{}wireless\PYGZhy{}essid ardrone2\PYGZus{}090332

auto wlan1
allow\PYGZhy{}hotplug wlan1
iface wlan1 inet dhcp
    pre\PYGZhy{}up wpa\PYGZus{}supplicant \PYGZhy{}B \PYGZhy{}Dwext \PYGZhy{}i wlan1 \PYGZhy{}c /etc/wpa\PYGZus{}supplicant/wpa\PYGZus{}supplicant.conf
    post\PYGZhy{}down killall \PYGZhy{}q wpa\PYGZus{}supplicant
\end{Verbatim}
\end{quote}

L'interface eth0 est configurée pour une connexion ethernet statique, le wlan0 pour se connecter au drone, et le wlan1
pour se connecter au wifi du campus grâce au fichier de configuration \code{wpa\_supplicant} que voici :
\begin{quote}
\end{quote}

A noter qu'à des fins de test, il est aussi possible de se connecter en réseau avec la raspberry pi en liaison ethernet
directe sans même avoir besoin de changer les fichiers de configuration présentés ci-dessus. C'est très utile lors de la
première configuration notamment. Pour celà, une fois une image système copiée sur la carte SD, il suffit de monter la
partition de la carte SD de 50mo ayant un système de fichier FAT/FAT32 (automatique sous Windows), puis d'éditer le
fichier \code{cmdline.txt} et d'ajouter à la fin de l'unique ligne du fichier : \code{ip=169.254.0.2}. Ensuite, lors du
prochain démarrage, la raspberry se configurera en adressage statique sur son interface ethernet, il suffit alors avec
son ordinateur de se mettre en adressage statique avec l'adresse \code{169.254.0.3} par exemple. Il est ensuite possible
de se connecter à la raspberry en ssh via
\begin{quote}

\begin{Verbatim}[commandchars=\\\{\}]
\PYGZdl{} ssh pi@169.254.0.2
\end{Verbatim}
\end{quote}


\subsection{Mails}
\label{raspberrypi:mails}
Dans la section démarrage, il était question d'envoyer un mail au démarrage de la raspberry pi contenant son adresse IP.
Les scripts disponible précédemment utilise la commande \code{mail}. On détail ici sa configuration. Tout d'abord, il faut
installer les paquets nécessaires :
\begin{quote}

\begin{Verbatim}[commandchars=\\\{\}]
\PYGZdl{} sudo apt\PYGZhy{}get install mailutils mpack
\PYGZdl{} sudo apt\PYGZhy{}get install ssmtp
\end{Verbatim}
\end{quote}

Puis d'ajouter la configuration de \code{ssmtp} dans le fichier \code{/etc/ssmtp/ssmtp.conf}. Voici le fichier utilisé cette année :
\begin{quote}

\begin{Verbatim}[commandchars=\\\{\}]
root=postmaster
mailhub=mail
hostname=raspberrypi

root=ddedale2016@gmail.com
mailhub=smtp.gmail.com:587
hostname=srvweb
AuthUser=ddedale2016@gmail.com
AuthPass=*******
FromLineOverride=YES
UseSTARTTLS=YES
\end{Verbatim}
\end{quote}

Il est ensuite possible d'envoyer un email de la façon suivante :
\begin{quote}

\begin{Verbatim}[commandchars=\\\{\}]
\PYGZdl{} \PYG{n+nb}{echo} \PYG{l+s+s2}{\PYGZdq{}Contenu du mail\PYGZdq{}} \PYG{p}{\textbar{}} mail \PYGZhy{}s \PYG{l+s+s2}{\PYGZdq{}Titre du mail\PYGZdq{}} destinataire@gmail.com
\end{Verbatim}
\end{quote}


\section{Liens externes}
\label{raspberrypi:liens-externes}
Une majorité des scripts inscrits dans cette page est disponible dans le dossier \code{raspberrypi} du dépot git : \href{https://github.com/DedaleTSP/Resources20152016}{Ici}


\chapter{ROS}
\label{ros::doc}\label{ros:ros}

\section{Introduction}
\label{ros:introduction}
ROS est un framework dédié à la robotique. Le choix d'utiliser ROS pour notre projet vient du fait qu'ayant du repartir
de 0 au mois de Mars, il nous fallait gagner du temps sur la réalisation de notre solution. L'intéret réside dans le fait
que pour ce framework existe un nombre important de paquets, notamment un driver pour l'ardrone et un paquet de
reconnaissance de tags. Nous n'avions donc plus à coder l'interface avec le matériel, et la reconnaissance de formes,
mais à les faire intéragir entre eux.


\section{Installation}
\label{ros:installation}

\subsection{Installer ROS}
\label{ros:installer-ros}
ROS peut être compliqué à installer à première vue, notamment lorsqu'on se lance dans son installation depuis les sources.
La méthode la plus simple et la plus sûr (perdre du temps sur cette étape n'est vraiment pas intéressant) consiste à
l'installer depuis des paquets binaires disponible dans les dépots des distributions officiellement supportées par ROS.
A ce jour, uniquement Ubuntu et ses variantes.
D'où l'utilité d'avoir une raspberry 2 ou plus pour pouvoir y installer le système \code{Ubuntu-ARM}.
ROS est disponible sous plusieurs versions. A ce jour, \code{ROS indigo} est la version avec le plus de compatibilité, et
est disponible sur \code{Ubuntu 14.04 (Desktop)} au maximum. Lors du choix de la version de ROS, il faut faire attention à
ce que tout les paquets ROS que l'on projette d'utiliser soit nativement compatibles avec cette versions (pour encore
une fois éviter les installations depuis les sources).

Plus de détails sur l'installation de ROS sur la documentation officielle \href{http://wiki.ros.org/fr/ROS/Installation}{ici}


\subsection{Installer un paquet ROS depuis les dépots Ubuntu}
\label{ros:installer-un-paquet-ros-depuis-les-depots-ubuntu}
Pour tous les paquets, la procédure d'installation est indiquée dans la documentation. Lorsque le paquet est disponible
sur les dépots Ubuntu, il suffit pour l'installer de lancer :
\begin{quote}

\begin{Verbatim}[commandchars=\\\{\}]
\PYGZdl{} sudo apt\PYGZhy{}get install ros\PYGZhy{}indigo\PYGZhy{}ardrone\PYGZhy{}autonomy
\end{Verbatim}
\end{quote}


\subsection{Créer un espace de travail}
\label{ros:creer-un-espace-de-travail}
Un espace de travail ROS est nécéssaire au développement de paquets. Pour initialiser un espace de travail, la démarche
est disponible sur la documentation officielle \href{http://wiki.ros.org/catkin/Tutorials/create\_a\_workspace}{ici}


\subsection{Installer un paquet depuis les sources}
\label{ros:installer-un-paquet-depuis-les-sources}
Pour installer un paquet depuis les sources, il faut charger les sources dans le répertoire \code{catkin\_ws/src/} puis
la depuis le répertoire \code{catkin\_ws} lancer les commande
\begin{quote}

\begin{Verbatim}[commandchars=\\\{\}]
\PYGZdl{} catkin\PYGZus{}make
\PYGZdl{} \PYG{n+nb}{source} \PYGZti{}/catkin\PYGZus{}ws/devel/setup.bash
\end{Verbatim}
\end{quote}


\section{Comprendre ROS}
\label{ros:comprendre-ros}
ROS utilise une structure particulière qui lui est propre. Un programme est constitué de \code{noeuds} qui s'échangent des
informations \code{messages} à travers des \code{topics}.
Un programme peut être lancé via un fichier de lancement \code{launch} ou directement en lançant le noeud (archaïque).
Différents outils en ligne de commande permettent de manipuler ROS, notamment \code{roslaunch} qui permet de lancer un fichier
\code{launch}, et \code{rostopic} qui permet d'intéragir directement avec les topics ouvert par un noeud (leur envoyer des messages
ou recevoir ce qu'ils émettent).
Pour plus d'informations, utilisez la commande \code{man}


\subsection{Lancer un fichier de démarrage}
\label{ros:lancer-un-fichier-de-demarrage}
Une fois un paquet installé, les fichiers de démmarage sont disponible dans son répertoire racine sous \code{/launch}.
Pour executer une fichier de démarrage :
\begin{quote}

\begin{Verbatim}[commandchars=\\\{\}]
\PYGZdl{} roslaunch \PYG{o}{[}PackageName\PYG{o}{]} \PYG{o}{[}LaunchFile\PYG{o}{]}
\end{Verbatim}
\end{quote}

Par exemple :
\begin{quote}

\begin{Verbatim}[commandchars=\\\{\}]
\PYGZdl{} roslaunch ardrone\PYGZus{}autonomy ardrone\PYGZus{}driver.launch
\end{Verbatim}
\end{quote}


\section{Utilisation du paquet ROS 2015-2016}
\label{ros:utilisation-du-paquet-ros-2015-2016}
Le paquet ROS écrit en 2015-2016 peut être installé et utilisé suivant la procédure (nécessite ROS indigo d'installé et
un espace de travail catkin sous \code{\textasciitilde{}/catkin\_ws/}
\begin{description}
\item[{\textbf{Installer les dépendances}}] \leavevmode
\item[{\textbf{Installer le paquet}}] \leavevmode
\item[{\textbf{Lancer le programme}}] \leavevmode
\end{description}


\section{Développement}
\label{ros:developpement}
Pour le développement et l'utilisation générale de ROS, la documentation officielle est très bien faite : \href{http://wiki.ros.org/fr/ROS/Tutorials}{ici}.
Une grande aide pour le développement est la consultation d'exemples dans les sources d'autres paquets.
\begin{itemize}
\item {} 
\href{https://github.com/FalkorSystems/falkor\_ardrone}{falkor\_ardrone}

\item {} 
\href{http://www.ros.org/wiki/tum\_ardrone}{tum\_ardrone}

\item {} 
\href{https://github.com/parcon/arl\_ardrone\_examples}{arl\_ardrone\_examples}

\item {} 
\href{https://github.com/mikehamer/ardrone\_tutorials}{AR Drone Tutorials}

\item {} 
\href{http://wiki.ros.org/tum\_simulator}{tum\_simulator}

\end{itemize}

Enfin, la documentation du paquet \code{ardrone\_autonomy} est très complète et disponible \href{http://ardrone-autonomy.readthedocs.io/}{ici}


\section{Liens externes}
\label{ros:liens-externes}
Installer ROS sur Ubuntu : \href{http://wiki.ros.org/fr/ROS/Installation}{ici}
Initialiser un espace de travail : \href{http://wiki.ros.org/catkin/Tutorials/create\_a\_workspace}{ici}
Tuto ROS : \href{http://wiki.ros.org/fr/ROS/Tutorials}{ici}


\chapter{Outils}
\label{tools::doc}\label{tools:outils}

\section{Git}
\label{tools:git}
Les bases de git (utile pour le travail collaboratif):
\begin{quote}

\begin{Verbatim}[commandchars=\\\{\}]
\PYGZdl{} git add \PYG{o}{[}Fichier\PYG{p}{\textbar{}}Dossier\PYG{o}{]}
\PYGZdl{} git commit \PYGZhy{}m \PYG{o}{[}Message\PYG{o}{]}
\PYGZdl{} git push
\end{Verbatim}
\end{quote}

La première commande sert à spécifier les fichiers ayant été modifiés, la deuxième à annoter leur modification, la
troisième à envoyer les modifications sur les serveurs distants. Ces commandes doivent être effectués dans un dossier
ayant été initialisé avec la commande \code{git clone}. Plus d'informations disponibles avec \code{man git}.

\textbf{Liens externes}

\href{https://github.com/DedaleTSP/}{Dépots Git 2015-2016}
\href{https://github.com/qnope/D-dale}{Dépots Git 2014-2015}
\href{http://www.upsilonaudio.com/category/drones/}{Upsilon Audio} - blog d'un ancien regroupant les premiers travaux sur le drone.



\renewcommand{\indexname}{Index}
\printindex
\end{document}
